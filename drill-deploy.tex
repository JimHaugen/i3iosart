%!TEX TS-program = pdflatex
%!TEX root = i3det-top.tex
%!TEX encoding = UTF-8 Unicode

\section{Drilling and Deployment}
IceCube transformed a cubic kilometer of Antarctic ice into an astrophysical particle detector composed of 86 cables (strings) of DOMs buried deep beneath the surface.  Each string required drilling a borehole approximately 60 cm in diameter to a depth of 2,500 m.  The 5 MW Enhanced Hot Water Drill (EHWD) was designed and built specifically for this task, capable of producing the required boreholes at a rate of one hole per 48 hours. Hot water drilling on this scale presented unique challenges that were successfully met with the EHWD.

Large-access holes were required at a high production rate.  Hot water drilling was selected as the best method due to its inherent speed.  Indeed, hot water drilling is the only feasible technology to provide rapid access to the deep ice on this scale.  Additionally, leaving the boreholes water-filled during drilling allowed for the deployed instrumentation to become frozen in place and optically coupled with the surrounding ice sheet.
 
Hot water drilling in this case involved a drilling phase to create the initial hole, followed by an upward reaming phase to give the hole a targeted diameter profile.  Hole diameter was oversized to compensate for closure from freezeback to provide sufficient time to deploy instrumentation, with contingency time for delays.  The elapsed duration from the end of drilling until the hole closes to below specification is referred to as the hole lifetime.  Substantial resources were invested in modeling the thermodynamics and shape of the hole over time to optimize hole lifetime and fuel consumption (Greenler, in press).

The Enhanced Hot Water Drill was designed and built to accomplish this task, and was continually refined over the course of IceCube construction.  At the project’s end, the EHWD had drilled 86 holes, each nominally 60 cm in diameter and 2,500 m deep, in seven field seasons (approximately 21 months total time).  Peak performance occurred in the 2009-2010 season with 20 holes drilled (early November to mid-January).  

The highest-level EHWD system design requirements were to:

1.	Deliver 80 boreholes, each 60 cm diameter and 2,500 m deep (actually delivered 86 holes),
2.	complete drilling and instrumentation deployment in seven field seasons (2004-2011),
3.	withstand the South Pole environment (average austral summer temperature -33°C, winter storage minimum temperature -80°C, altitude approximately 3,000 m),
4.	stay compatible with South Pole logistics (all large cargo and fuel transported by LC-130 aircraft),
5.	simultaneously support deployment of in-ice instrumentation (to streamline the drilling-deployment flow),
6.	minimize drill time and fuel consumption, and
7.	maintain safe and predictable operations.

The design philosophy for the EHWD was to leverage and build upon the drilling experiences of AMANDA (Antarctic Muon and Neutrino Detector Array – Koci, 1994; AMANDA Collaboration, 2001; Koci, 2002), the prototype detector that served as a proof of principal for IceCube.  This was accomplished by reusing equipment where appropriate, recruiting expertise, and incorporating the following major enhancements:

1.	Doubled thermal capacity (2.3 MW to 4.7 MW),
2.	continuous drill hose on a single hose reel (eliminating the need to add/remove hose segments throughout drilling/reaming),
3.	extensive system automation,
4.	two drilling structures (allowing for streamlined drilling-deployment flow),
5.	modular design,
6.	high-efficiency water heaters (to reduce fuel consumption), and
7.	improved drilling strategy (optimizing hole shape to avoid over-drilling the borehole diameter and wasting fuel).


The EHWD system was implemented across two separate sites.  The seasonal equipment site (SES) provided electricity and a stable supply of hot pressurized water, and the tower operations site (TOS) was where the hole was drilled.  The two sites were linked by long cables and insulated hoses.

The SES was comprised of generators, water tanks, pump and heating buildings, a central control building, mechanical and electrical shops, spare parts storage, and system Rodwell.  All of the equipment was packaged into customized ISO shipping containers that fit into an LC-130 aircraft and tailored to its specific building functions.  These packaged containers were called mobile drilling structures (MDS).  Hoses and cables connected SES subsystem buildings together, and wherever possible custom electrically heated hoses were installed, providing an effective freeze mitigation strategy.

The TOS included the drill tower and attached operations building as well as the hose and cable reels.  There were two towers and one set of drill reels.  After drilling, drill reels were moved to the next hole location, where the second tower had already been staged.  The first tower stayed at its existing location to support deployment of the instrumentation.  Once deployment had finished, the first tower could be moved to the next location while drilling at the second tower was underway.  This leapfrog sequence of the tower structures reduced hole turnover time and allowed for nearly continuous drilling operations.

Due to the massive size and complexity of the SES, it remained stationary throughout each drill season.  At the end of the drill season, the SES was decommissioned and repositioned within a virgin sector of the IceCube array, staged for the following drilling season.  The distance between the SES and TOS had a practical limit, referred to as reach, which defined the boundary of a seasonal drilling sector.  Reach of the EHWD was 450 m, limited by pressure and voltage drop through the SES-TOS link.  During the last two seasons of IceCube construction the SES supported two drilling sectors from the same SES location, demonstrating a reach of 430 m.

Each drilling season started with a staggered arrival of drill crew members while the SES and TOS were excavated and commissioned.  Season startup tasks included SES and TOS warming and hookups, reinstallation of do-not-freeze equipment (such as motor drives, sensors, and some rubber/gaskets), generator commissioning, safety checkout and system tuning, seed water delivery, and Rodwell development.  This phase typically took four weeks, three of which had a full crew of approximately 30 drillers.  Once seed water was delivered, the transition was made to around-the-clock operations spread across three 9-hour shifts.

The production drilling sequence was to drill, ream, and move to the next location.  Independent firn drilling stayed ahead of deep drilling by no fewer than a couple of holes, and often the Rodwell and first few holes of the season had already been firn-drilled the prior season.  The phase between holes, called idle, was characterized by minimal flow through the system and fine-tuning of Rodwell management strategies.  The idle phase also included regular maintenance tasks and deployment of IceCube instrumentation.  Hole production rate was 48 hours per hole on average, and the quickest cycle time was 32 hours.

System shutdown would begin approximately two weeks before season’s end.  Shut down tasks included flushing the system with propylene glycol, blowing out the plumbing with compressed air, removing do-not-freeze equipment for warm storage, storing the TOS structures and other support equipment, and finally moving the SES into location for the following season.

During steady drilling operations, a minimum of four people were needed to operate the drill (drill control center, SES float, TOS control, TOS float), but a full shift was required to move hole locations and aid in deployment.  Minimum shift size was nine team members – this allowed for one sick worker and a 4/4 split lunch.  A critical part of the staffing plan was to have a good spread of expertise on all shifts, including a shift lead and deputy lead comfortable with all aspects of the system and its operation, a safety officer (duty of the deputy lead), an electronics expert, a heater expert, a software expert, and mechanically proficient technicians.  The most important thing IceCube did to assure successful drilling operations was put a strong focus on retention of experienced drillers and talent.

A strong safety culture was an essential aspect of drilling and deployment operations.  Annually-reviewed safety processes included a safety manual, 34 standard operating procedures, 18 hazard analyses, and a series of checklists.  In the field, safety briefs, emergency management exercises, near-miss incident reporting, and peer safety audits became matter of course.  Off the ice, a 2-week driller training course was organized each year to highlight system technical updates, discuss plans for the coming season, and provide further safety and first aid training.  As a result of this safety culture, IceCube had only 4 lost time drilling-related safety incidents in approximately 52 on-ice person-years.

EHWD System Characteristics. (This is a table that needs to be formatted)
Specification
Value
Total Power/Thermal/Electrical
5/4.7/0.3 MW
Maximum Drill Speed
2.2 m/min
Maximum Ream Speed
10 m/min
Water Flow (delivered to main hose)
760 l/min
Water Temperature (delivered to main hose)
88 °C
Water Gauge Pressure (at main pumps)
7,600 kPa
Average performance for 24-hour lifetime hole1
2500 m depth
    Total Fuel2, AN-8
21,000 L
    Time to Drill/Ream
30 hr
    Hole Production Cycle Time3
48 hr
Observed peak performance for 24-hour lifetime hole1,4
2500 m depth
    Total Fuel2, AN-8
15,000 L
    Time to Drill/Ream
27 hr
    Hole Production Cycle Time3
32 hr
1.	Hole diameter remains greater than 45 cm for 24 hours after completion of drilling.
2.	Total Fuel includes deep drilling/reaming and firn drilling.
3.	Hole Production Cycle Time is the elapsed time from start of one hole to start of the next hole.
4.	IceCube hole 32.

With the EHWD, we have demonstrated that is it possible to do large-scale production ice drilling in the Antarctic environment in a safe, efficient, and predictable way.  Critical components of IceCube’s successful drilling campaign included a large engineering investment, steadfast year-to-year support to properly address lessons learned, a strong safety culture, and priority on retention of experienced crew members.
